LTspice is a circuit simulation software, where circuits can be modeled, parametrized and simulated.
A modeled circuit is stored in a LTspice schematic file, with the file extension .asc.
Since, the last step in the pipeline of this thesis is the conversion of the hand drawn \acp{ECD} into a LTspice schematic file, the file structure and syntax were analyzed and are presented in this section.

LTspice files are written in plain text and are human readable.
The file structure can be interpreted line by line, meaning that one command is written on one line.
Inside a line a command is separated by a space.
In most cases the first word of a line is a keyword indicating the used command, which then is followed by parameters for the command.

LTspice itself provides a grid, where components are aligned to.
This grid has a size of $32x32$ units, where a unit is an abstract measure inside of LTspice.

\subsection{LTspice Schematic File Syntax}

\subsubsection{Header}

Each schematic file starts with a header, which defines the used version of the syntax.
Throughout this thesis the forth version of the syntax is used. Table \ref{tab:ltheader_syntax} shows the syntax for the version definition command.

\begin{table}[H]
\begin{center}

\begin{tabular}{l|l|l}
    & \textbf{Keyword} & \textbf{Param1}\\
    \hline
    \textbf{Syntax} & VERSION & version number\\
    \textbf{Info} & & in this thesis 4
\end{tabular}
\caption{LTspice header syntax}
\label{tab:ltheader_syntax}

\end{center}
\end{table}


\subsubsection{Symbols}

In LTspice \acp{ECC} are called symbols.
The syntax to define a symbol is presented in table \ref{tab:ltsymbol_syntax}.
The command is declared by using the keyword \textit{SYMBOL} followed by a symbol name, where the symbol name is a mapping to an \ac{ECC}.
All symbol names which are used in this thesis are shown in table \ref{tab:ltsymbol_mapping}.
The symbol name is followed by two integers defining the $x$ and $y$ coordinate of the symbol.
The coordinates are representing the upper left corner of the symbols image used to represent the symbol inside LTspice.
Additionally a rotation has to be provided with $Rr$ where $r$ defines the rotation in degree.
The rotation $r$ is constrained to be either $0$, $90$ or $270$ degree.
So an example for a resistor declaration would be: \textit{SYMBOL res 32 32 R90}, which means that a resistor is defined at $x = 32$, $y = 32$ with a rotation of $90$ degree.

\begin{table}[H]
\begin{center}

\begin{tabular}{l|l}
    \textbf{Electrical Circuit Component} & \textbf{LTspice keyword}\\
    \hline
    Resistor & res\\
    Capacitor & cap\\
    Inductor & ind\\
    Diode & diode\\
    Voltage Source & voltage\\
    Current Source & current
\end{tabular}
\caption{LTspice symbol names}
\label{tab:ltsymbol_mapping}

\end{center}
\end{table}

\begin{table}[H]
\begin{center}

\begin{tabular}{l|l|l|l|l|l}
    & \textbf{Keyword} & \textbf{Param1} & \textbf{Param2} & \textbf{Param3} & \textbf{Param4}\\
    \hline
    \textbf{Syntax} & SYMBOL & symbol name & X-Coordinate & Y-Coordinate & Rotation\\
    \textbf{Info} & & see table \ref{tab:ltsymbol_mapping} & multiple of 32 & multiple of 32 & R0, R90, R270
\end{tabular}
\caption{LTspice symbol syntax}
\label{tab:ltsymbol_syntax}

\end{center}
\end{table}

\subsubsection{Symbol Attributes}

A symbol can be further specified through the usage of a symbol attribute.
The symbol attribute is always applied to the first occurrence of a previously defined symbol relative to this command.
The syntax for this command is presented in table \ref{tab:ltsymattr_syntax}.
This command is declared using the keyword \textit{SYMATTR} followed by the targeted attribute and the corresponding value to be set for the targeted attribute.
Two attributes can be used as a target for this command.
The \textit{InstName} attribute allows to declare a name for the symbol and the value therefore is a string.
The \textit{Value} attribute allows to declare a component value for the symbol always defined in the corresponding base unit of the component (e.g. $\Omega$ for resistors).

\begin{table}[H]
\begin{center}

\begin{tabular}{l|l|l|l|l|l}
    & \textbf{Keyword} & \textbf{Param1} & \textbf{Param2}\\
    \hline
    \textbf{Syntax} & SYMATTR & attribute & value\\
    \textbf{Info} & & Value, InstName & Integer, String\\
\end{tabular}
\caption{LTspice symbol attribute syntax}
\label{tab:ltsymattr_syntax}

\end{center}
\end{table}

\subsubsection{Ground}

Grounds are defined by using the keyword \textit{FLAG} followed by the coordinates of the ground.
An additional \textit{0} has to be placed at the end of the line, which indicates that the used flag is indeed a ground node.
Note that it was not further analyzed which effect the last parameter has on the flag definition, but it can be said that when the \textit{0} is not present the ground is not defined correctly.
The syntax for grounds can be seen in table \ref{tab:ltflag_syntax}.

\begin{table}[H]
\begin{center}

\begin{tabular}{l|l|l|l|l}
    & \textbf{Keyword} & \textbf{Param1} & \textbf{Param2} & \textbf{Param3}\\
    \hline
    \textbf{Syntax} & FLAG & X-coordinate & Y-coordinate & flag indicator\\
    \textbf{Info} & & multiple of 32 & multiple of 32 & 0 for ground\\
\end{tabular}
\caption{LTspice ground syntax}
\label{tab:ltflag_syntax}

\end{center}
\end{table}

\subsubsection{Wire}

After the symbols have been defined they are connected through wires.
Wires in LTspice are defined as lines.
The syntax is presented in table \ref{tab:ltwire_syntax}.
The command begins with the keyword \textit{WIRE} followed by two coordinate pairs.
The first pair is the beginning of the line and second the end of the line.

\begin{table}[H]
\begin{center}

\begin{tabular}{l|l|l|l|l|l}
    & \textbf{Keyword} & \textbf{Param1} & \textbf{Param2} & \textbf{Param3} & \textbf{Param4}\\
    \hline
    \textbf{Syntax} & WIRE & X1-coordinate & Y1-coordinate & X2-coordinate & Y2-coordinate\\
    \textbf{Info} & & multiple of 32 & multiple of 32 & multiple of 32& multiple of 32  \\
\end{tabular}
\caption{LTspice wire syntax}
\label{tab:ltwire_syntax}

\end{center}
\end{table}
