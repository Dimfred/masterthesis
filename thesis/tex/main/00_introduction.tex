\chapter{Introduction}

\section{Motivation}

\section{Related Works}

The aim of this thesis is to provide a conversion pipeline for hand-drawn \acp{ECD} into the LTspice schematic file format.
To provide some context for the task, the following subsections present approaches related to the goal of this thesis.

- TODO general structure

\subsection{Classification of Eletrical Circuit Components}

G"unay et al. \cite{ecd_basecnn} compared in their work the accuracy of various common \ac{CNN} architectures trained on a dataset of hand-drawn \acp{ECC}.
The dataset consisted of four classes (resistor, capacitor, inductor, voltage source) and had overall a size of 863 images. best cnn 83\%


% -- Classification only

% - base_cnn

% - anngeo
% - letters, numbers and 7 symbols 20 / class
% - different moment features
% - fed into ann

% - texturesmo
% - 20 eccs 150 samples / class
% - hog, texture based features optimized with ReliefF
% - fed in to SMO (Sequential minimal optimization)
% - fourier svm

% - ctxindependentsvm

% -- Object Detection

% - logical gates (AND, OR, NOT)
% - segmentation of components based on closed region
% - draw bbox around region
% - feature extraction + svm
% - 180 samples
% - 83 \% accuracy

% - knn recog
% - full circuits, with 4 different classes
% - 100 images of circuits
% - segmentation of components
% - knn classification based on features (area, major / minor axis, centroid, orientation
%     eccentricity)
% - 90\% on training set

% - seghogsvm
% - 10 classes
% - 35 components / class
% - 350 components: 200 train / 150 test
% - segmentation based on closed shape, connected lines, disconnected lines
% - hog features
% - svm
% - 87.71\% accuracy (including segmentation)


% -- FULL CONV

% - knnfull

% - yolobool

% -- Online Recognition

% - hmm

% - hmm2



\section{Goals of the Thesis}

\subsection{Task Description}

\subsection{Contribution}
