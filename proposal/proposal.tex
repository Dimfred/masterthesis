%macros
\newcommand{\lf}{\\[\baselineskip]}
\documentclass{article} % use option titlepage to get the title on a page of its own.

%%%%%%%%%%%%%%%%%%%%%%%%%%%%%%%%%%%%%%%
% Packages

\usepackage{fancyhdr}
\usepackage
[
        a4paper,% other options: a3paper, a5paper, etc
        left=2cm,
        right=2cm,
        top=2.5cm,
        bottom=2cm,
        % use vmargin=2cm to make vertical margins equal to 2cm.
        % us  hmargin=3cm to make horizontal margins equal to 3cm.
        % use margin=3cm to make all margins  equal to 3cm.
]
{geometry}


%%%%%%%%%%%%%%%%%%%%%%%%%%%%%%%%%%%%%%%
% Other Stuff

% header
\pagestyle{fancy}
\fancyhf{}
\lhead{Master Thesis}
\rhead{Dmitrij Vinokour}

%%%%%%%%%%%%%%%%%%%%%%%%%%%%%%%%%%%%%%%
% Document


\begin{document}
\begin{center}

\textbf{
\huge{Conversion of Hand Drawn Electrical Circuit Schematics into LTSpice Format using Deep Learning Methods}
\\~\\
\large{Konvertierung von Handgeschriebenen Elektrischen Schaltungen in LTSpice Format mithilfe von Deep Learning Methoden}
\\~\\
Dmitrij Vinokour
\\~\\
}
Pattern Recognition Lab, Friedrich-Alexander-University, Erlangen-Nuremberg
\\
Supervisors: Florian Thamm M.Sc., Felix Denzinger M.Sc., Prof. Dr. Andreas Maier
\\~\\
\noindent\rule{\textwidth}{1pt}
\end{center}
\section*{Thesis Description}
An electrical circuit schematic (ECS) is a way to represent an electrical circuit in a formal way. ECSs consist of symbols representing electrical circuit components (ECC) and wires (lines) connecting those components. Additionally an ECS can contain component annotations describing the name and value of a component. As an example, an annotation for a resistor could be "R0 1000 $\Omega$" where "R0" represents the name of the resistor and "1000 $\Omega$" its component value.
\lf
The knowledge of the schematic symbols and the corresponding calculations to acquire different values, such as voltage,  current  or component values is taught in school and university [LEHRPLAN]. To verify results of performed calculations Students rely on solutions, or on dedicated circuit simulation software such as LTSpice. In the latter case the hand drawn electrical circuit has to be rebuild in the application. Application based approaches are considered to be ~90\% more expensive, than hand drawn methods \cite{ctxindependentsvm} and also require a priori knowledge of the underlying program. Hence an automated method to convert a scan of an ECS into a digital format, recognizable by a simulation software, would greatly benefit the result verification efficiency.
\lf
So far various researches have been conducted on the segmentation, recognition and the tracing of inter ECC connections. The proposed approaches can be structured in the following way. 1) Classification of already segmented ECCs \cite{anngeo, basecnn, texturesmo}. 2) Segmentation and classification of ECCs \cite{seghogsvm, fouriersvm}. 3) Segmentation and classification of ECCs and tracing of the wire to acquire the underlying ECS topology \cite{knnfull}.
\lf
\begin{enumerate}
\item Object detection of ECCs in a scan of a circuit (YoloV4-tiny)
\item Segmentation of the circuit from the gridded / non-gridded paper (Mobile-UNet)
\item Tracing inter ECC connections, and building topology (Connected Components, Breadth-first search)
\end{enumerate}
%Each symbol in corresponds to a hardware component in the TODO "real world". Those hardware components have a component value associated to them. With that TODO UEBERBGRIFF "values" like voltage, current and other "values" can be calculated to analyze the circuit TODO MEH. The symbols (TODO for schematics) and (TODO the formulas to perform) calculations to obtain component values are thought in school and university. (TODO) To verify the validity of their calculations, students rely on solutions or on the verification from dedicated circuit simulation programs like LTSpice. (TODO) To obtain verification results from LTSpice the hand drawn circuit has to be rebuild in the software, that produces overhead which could be evaded when the scan of the hand drawn circuit could be directly converted into LTSpice format.

% \par\mbox{}
%The conversion problem can be decomposed in TODO NUMBER problems.
% \par\mbox{}
%Despite the growing digitalization hand drawn electrical circuit schematics are still thought in school and university. Various calculations can be performed to identify values like voltage, current or component value or current flowing through those components. Often without having the solution for the calculation students are not able to verify their results. It is although possible to redraw the circuit in dedicated circuit simulation tools to obtain the right values for the calculation. This requires knowledge of the
\bibliographystyle{unsrt}
\bibliography{proposal}
\end{document}
